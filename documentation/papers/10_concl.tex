\section{Conclusions and Future Work}
\label{concl}
In this paper we have presented Heimdall, a system that is capable of detecting security vulnerabilities on a mobile device. We use a heuristics based approach to detect these vulnerabilities. We also discussed a potential loophole in Android's custom content provider's security. This loophole could allow a malicious app to steal users' data from their phones. We built a proof-of-concept for demonstrating this loophole and we also tested a random sample of 1500 apps to find out if this vulnerability exists in them. We reverse engineered two popular apps and a non-popular app, to see if there are additional checks present in the code, to handle access control to data. We presented our observations made in our evaluation process which led us to conclude that such checks are possible and are present in some apps but not all apps. In conclusion we can say that this is a potential loophole that could lead to user data leakage and thus have serious implications. Therefore, there needs to be some checking mechanism in form of an API to verify app signature keys or to verify component access control or maybe even strict permission association for custom providers.

In future, we hope to include more heuristics in Heimdall and capture more such vulnerabilities. As we have discussed in the related work section, a lot of research has been undertaken in this area and we hope to incorporate the ability to detect these vulnerabilities in Heimdall. Since the Heimdall project's primary goal is to be deployed in a BYOD scenario we are working on a mechanism to actually control the data flow on Android. This will allow us to study what data is being transferred to and from the phone as well as implement policies defined by the system administrator. Detecting discrepancy between app's expected behavior and actual behavior is also being studied by many researcher but we feel this problem still remains unsolved and we hope to make that into a feature of Heimdall. 

\section{Acknowledgments}
Support for this work was provided by NSF grants 0910838 and 1228198.
% The following two commands are all you need in the
% initial runs of your .tex file to
% produce the bibliography for the citations in your paper.
\bibliographystyle{abbrv}
\bibliography{bibliography}  % sigproc.bib is the name of the Bibliography in this case

% You must have a proper ".bib" file
%  and remember to run:
% latex bibtex latex latex
% to resolve all references
%
% ACM needs 'a single self-contained file'!
%
% That's all folks!

\end{document}