%{\Large %Remove comment '%' to get a 'Large' font introduction section and abstract
\begin{abstract}
The number of mobile devices in the world surpassed the number of personal computers in 2010. Mobile devices now carry sensitive personal data, captured through sensors on the phone, as well as confidential corporate data through work emails and apps. As a result, they have become lucrative targets for attackers and the privacy and security of these devices have become a vital issue. Existing access control mechanisms on these devices, which mostly rely on a one-time permission grant, are too restrictive and inadequate. Such mechanisms are incapable of controlling contextual or custom app-data flows. In this paper we focus on this scenario and show how data leakages may occur due to developer inadequacy and a lack of proper checks for such leakages. We describe a potential loophole in the Android permission verification mechanism and a way to capture such a vulnerability on a user's mobile device. We also show a mechanism of injecting such a vulnerability into any app.
\end{abstract}
% A category with the (minimum) three required fields
\category{D.4.6}{Operating Systems}{Security and Protection}[Access Controls]
%\terms{Access Control, Mobile privacy and security}
\keywords{Access Control, Android Content Providers, Permission Control}
