\section{Introduction}
\label{intro}
Smart phones have become ubiquitous due to its low cost and Android is the biggest player in the market. It boast of more than a billion active 30 day users according Engadget \citep{Engadget_market_share}, which comes down to 84.8\% market share in the smart phone category. The different apps running on Android make the phone a very powerful device. Apart from Googles dominant Google Play, a variety of other app-stores co-exist which include Amazon App Store, Slide Me, 1 Mobile Market, Samsung Galaxy Apps, Opera Mobile store etc \citep{Online_App_Stores}. According to Statista \citep{Android_app_number}, more than 1.6 million android application have been published for use. 

The proliferation of smart phones encouraged all the domains to be linked to the smart phone for easier and faster access. Medical domain is also not different and a number of hardware accessories had come up. Some of them include Blood glucose monitoring system[5], Blood pressure monitor[6], Heart monitor[7], Fitness accessories like Fitbit, Breathalyzer[4] etc. All these accessories have an associated Android application which interacts with the physical device, process the data, display it and stores relevant information. Another category of applications are those used by hospitals and its employees to manage information about the patients, doctors etc. The important point is that all these sensitive information are being received, processed, stored and possibly updated to cloud storage by these applications. 
The fact that personal and sensitive data is handled by these devices presents a potential security risk. With more than 1.6 million apps to choose from, let alone a layman user, even a computer expert will have difficulty to understand what are benign and what are malicious. One of the most traditional ways to look at this problem is to use signature based malware detection, in which once a malware application is detected, signature will be created once and the signatures are updated to a common database. Any future appearance of the same malware would be detected using these signatures. The classical problem with this approach is the occurrence of mutating malwares which changes it’s signature every time it is spread and it inability to detect future similar attacks. Apart from this Android apps has issues on its own. One of the main problem is the high volatility of the app markets. Since unlike Apple store, Google play store is not verified and hence new applications becomes available and taken out quite frequently. Yet another issue is repackaging popular application with malware. Often these repackaged applications are uploaded to other local App-stores where the original application is not available. 
Hence we require a faster and dynamic setup to detect unsafe or rather potentially unsafe applications. [Paper to be referred Dr B]Even though not totally accurate, there are some intuitions which we can utilize from the meta-data available with each application. The different meta-data available with each of these applications include App description, Permissions required, Number of downloads, Broad Application categories etc. As an example of our intuition let us take the case of “brightestlight” application, which is available in Google Play Store. The description of the application says that it is a flash light app, which shows some unobtrusive Ads. But on analyzing the Permissions required, we can see that the application requests permissions for location (accurate GPS and approximate network based), file manipulation (read, modify and delete USB storage), camera (curiously it includes auto-focus permission) etc. This does seem suspicious. But still we cannot say that the application is unsafe, rather we need further investigation. 
\section{Related Work}
\label{RelatedWork}
\begin{comment}
Playdrone : Crawls Playstore- how playstore evolved - source code analysis of library usage - similar app detection-secret authentication key storage (can be found by decompilation)
(1) native libraries are heavily used by popular Android applications, limiting the benets of Java portability and the ability of Android server ooading systems to run these applications, (2) 25%
of Google Play is duplicative application content, and (3) Android applications contain thousands of leaked secret au-thentication keys which

Andradar : First, we can discover malicious applications in alternative markets, second, we can expose app distribution strategies used by malware developers, and third, we can monitor how different markets react to new malware. 
to identify and track malicious apps still available in a number of alternative app markets.

Android Security : discuss the Android security enforcement mechanisms,
threats to the existing security enforcements and related issues,
malware growth timeline between 2010 and 2014, and stealth
techniques employed by the malware authors, in addition to the
existing detection methods. This review gives an insight into the
strengths and shortcomings of the known research methodologies
and provides a platform, to the researchers and practitioners,
toward proposing the next-generation Android security, analysis,
and malware detection techniques.

ANDRUBIS:, a fully automated,
publicly available and comprehensive analysis system for
Android apps. ANDRUBIS combines static analysis with dynamic
analysis on both Dalvik VM and system level, as well as
several stimulation techniques to increase code coverage.

changes in the malware threat landscape
and trends amongst goodware developers. Dynamic code
loading, previously used as an indicator for malicious behavior,
is especially gaining popularity amongst goodware

App analysis for astma!!

App behavoir against description
CHABADA tool
clustering apps by description topics, and identifying outliers
by API usage within each cluster, our CHABADA approach effectively
identifies applications whose behavior would be unexpected
given their description.
Recommendations for android eco system

\end{commit}
