\section{Related Work}
\label{RelatedWork}
Research being done to predict user's preferences by a number of people~\cite{Benisch2011,Sadeh2009,lin2014soups,liu2014www}. Owing to that research we make an assumption that it is possible to fairly accurately create user permission choices on Android devices. However, our goal is separate from theirs in a threefold manner. Firstly we are defining policy rules for users which may allow, deny or allow with caveat specific permissions depending on the user context. Secondly we are not trying to show that it is possible to learn a user's policy from scratch rather we are agreeing with their observation that it is possible to use privacy profiles to define or group user preferences~\cite{liu2014www}. Instead we are trying to show that with user feedback it is possible to reach an individual user's ``perfect'' policy with a certain probability. Thirdly, we are researching ways to include app provenance information, api usage and observed mobile behavior~\cite{enck2010taintdroid} to compute a metric that will accurately measure the trustworthiness of an app.
\hl{
Playdrone : Crawls Playstore- how playstore evolved - source code analysis of library usage - similar app detection-secret authentication key storage (can be found by decompilation)
(1) native libraries are heavily used by popular Android applications, limiting the benefits of Java portability and the ability of Android server overloading systems to run these applications, (2) 25\% of Google Play is duplicative application content, and (3) Android applications contain thousands of leaked secret authentication keys which

Andradar : First, we can discover malicious applications in alternative markets, second, we can expose app distribution strategies used by malware developers, and third, we can monitor how different markets react to new malware. To identify and track malicious apps still available in a number of alternative app markets.

Android Security : discuss the Android security enforcement mechanisms, threats to the existing security enforcements and related issues, malware growth timeline between 2010 and 2014, and stealth techniques employed by the malware authors, in addition to the existing detection methods. This review gives an insight into the strengths and shortcomings of the known research methodologies and provides a platform, to the researchers and practitioners, toward proposing the next-generation Android security, analysis, and malware detection techniques.

ANDRUBIS:, a fully automated, publicly available and comprehensive analysis system for Android apps. ANDRUBIS combines static analysis with dynamic analysis on both Dalvik VM and system level, as well as several stimulation techniques to increase code coverage. 

changes in the malware threat landscape and trends amongst goodware developers. Dynamic code loading, previously used as an indicator for malicious behavior, is especially gaining popularity amongst goodware App analysis for astma!!

App behavoir against description CHABADA tool clustering apps by description topics, and identifying outliers
by API usage within each cluster, our CHABADA approach effectively
identifies applications whose behavior would be unexpected
given their description.
Recommendations for android eco system}

author et.al. developed a formal android permission model to analyze the permission protocol used using Alloy. Using Alloy analyzer, they reasoned over the model to detect possible vulnerabilities in the protocol. It also generated counter examples which can possibly exploit the vulnerabilities in the protocol. Their work could detect a vulnerability in which an application with normal security level permission was able to access another application's dangerous level custom permission given the following conditions. First, both the permission names are the same and second the application with lesser security level is installed first. But we differentiate from them such that instead on just focusing on