\section{Related Work}
\label{RelatedWork}
\noindent
A lot of work has been done in trying to detect malicious apps and malicious behaviors on mobile devices before. Techniques have included detecting re-delegation of permissions studied by Felt et.al.~\cite{Felt11permissionReDelegation} in which an app with higher privileges performs tasks for another app with less privileges. A study by Zhou et.al.~\cite{Zhou2012MalwareGenomeProject} have looked at ways for detecting malware by studying their characteristics like installation methods, activation mechanisms and malicious payload nature. The WHYPER framework was created by Pandita et. al.~\cite{Pandita2013Whyper}, that used Natural Language Processing (NLP) techniques to identify the need for a permission in an app by reading the app's description. Our work focuses on using the knowledge from all these studies and incorporating them as heuristics for Heimdall to detect.

There have been multiple attempts at achieving the goal of properly managing access control on mobile (Android) devices. Efforts have been made by the open source community through the XPrivacy project (needs a rooted phone), the Privacy Guard project (available on Cyanogenmod, a custom Android ROM), the PDroid application (needs a rooted device). Research project by Conti et. al.~\cite{conti2011crepe} (CRePe), Enck et al.~\cite{enck2010taintdroid} (TaintDroid) and Jagtap et al.~\cite{Jagtap2011Privacy} (Preserving Privacy in Context-Aware Systems) have made similar efforts. CRePE described a system where security policy enforcement was carried out based on context of the smart phone. TaintDroid was a research effort where the data flow on an Android device was studied to figure out when sensitive data left the system via an untrusted application. The work of Jagtap et al.~\cite{Jagtap2011Privacy} focused on constraining data flow in a context-aware system using a policy-based framework. A related work by Ghosh et al.~\cite{ghosh2012privacy} used a similar policy driven approach to constrain application permissions based on context. In our work we are generating recommendations by studying app metadata and apks. We allow system administrators to choose from these recommendations and make them part of the corporate policy.

In a study carried out by Huckvale et. al.~\cite{raey} it was found that despite the tests carried out by National Health Service (England) to ensure clinical data safety standards, apps had flouted privacy standards and sent data without encryption. This is the reason we are focusing on the loophole in content provider permission and the lack of any standard mechanism to verify whether a requester has access permissions or not. Expecting that the app developer would ensure security and leaving this loophole in android exposes users' data to potential attacks.

