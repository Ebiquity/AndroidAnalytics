\section{Related Work}
\label{RelatedWork}
\noindent
There have been multiple attempts at achieving the goal of properly managing access control on mobile (Android) devices. Efforts have been made by the open source community through the XPrivacy project (needs a rooted phone), the Privacy Guard project (available on Cyanogenmod, a custom Android ROM), the PDroid application (needs a rooted device). Research project by Conti et. al.~\cite{conti2011crepe} (CRePe), Enck et al.~\cite{enck2010taintdroid} (TaintDroid) and Jagtap et al.~\cite{Jagtap2011Privacy} (Preserving Privacy in Context-Aware Systems) have made similar efforts. CRePE described a system where security policy enforcement was carried out based on context of the smart phone. TaintDroid was a research effort where the data flow on an Android device was studied to figure out when sensitive data left the system via an untrusted application. The work of Jagtap et al.~\cite{Jagtap2011Privacy} focused on constraining data flow in a context-aware system using a policy-based framework. A related work by Ghosh et al.~\cite{ghosh2012privacy} used a similar policy driven approach to constrain application permissions based on context.
\hl{
Playdrone : Crawls Playstore- how playstore evolved - source code analysis of library usage - similar app detection-secret authentication key storage (can be found by decompilation)
(1) native libraries are heavily used by popular Android applications, limiting the benefits of Java portability and the ability of Android server overloading systems to run these applications, (2) 25\% of Google Play is duplicative application content, and (3) Android applications contain thousands of leaked secret authentication keys which

Andradar : First, we can discover malicious applications in alternative markets, second, we can expose app distribution strategies used by malware developers, and third, we can monitor how different markets react to new malware. To identify and track malicious apps still available in a number of alternative app markets.

Android Security : discuss the Android security enforcement mechanisms, threats to the existing security enforcements and related issues, malware growth timeline between 2010 and 2014, and stealth techniques employed by the malware authors, in addition to the existing detection methods. This review gives an insight into the strengths and shortcomings of the known research methodologies and provides a platform, to the researchers and practitioners, toward proposing the next-generation Android security, analysis, and malware detection techniques.

ANDRUBIS:, a fully automated, publicly available and comprehensive analysis system for Android apps. ANDRUBIS combines static analysis with dynamic analysis on both Dalvik VM and system level, as well as several stimulation techniques to increase code coverage. 

changes in the malware threat landscape and trends amongst goodware developers. Dynamic code loading, previously used as an indicator for malicious behavior, is especially gaining popularity amongst goodware App analysis for astma!!

App behavior against description CHABADA tool clustering apps by description topics, and identifying outliers
by API usage within each cluster, our CHABADA approach effectively
identifies applications whose behavior would be unexpected
given their description.
Recommendations for the Android ecosystem}

author et.al. developed a formal Android permission model to analyze the permission protocol used using Alloy. Using Alloy analyzer, they reasoned over the model to detect possible vulnerabilities in the protocol. It also generated counter examples which can possibly exploit the vulnerabilities in the protocol. Their work could detect a vulnerability in which an application with normal security level permission was able to access another application's dangerous level custom permission given the following conditions. First, both the permission names are the same and second the application with lesser security level is installed first. But we differentiate from them such that instead on just focusing on
