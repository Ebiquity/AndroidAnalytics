\section{Related Work}
\label{RelatedWork}
\noindent
A lot of work has been done in trying to detect malicious apps and malicious behaviors on mobile devices before. A survey by Enck~\cite{Enck2011Defending} discuses most of the state-of-art in smartphone research, including efforts in designing new OS protection mechanisms, as well as performing security analysis of real apps. These techniques have included detecting re-delegation of permissions studied by Felt et.al.~\cite{Felt11permissionReDelegation} in which an app with higher privileges performs tasks for another app with less privileges. A study by Zhou et.al.~\cite{Zhou2012MalwareGenomeProject} have looked at ways for detecting malware by studying their characteristics like installation methods, activation mechanisms and malicious payload nature. Permission analysis research done by Barrera et al.~\cite{Barrera2010SOM} used Self Organizing Maps (SOM) to analyze permission usage. The WHYPER framework on the other hand tried to match purpose to permission, carried out by Pandita et. al.~\cite{Pandita2013Whyper} and used Natural Language Processing (NLP) techniques to identify the need for a permission in an app by reading the app's description. In another work done by Gorla et. al.~\cite{GorlaCheckingAppBehavior2014} apps were clustered by descriptions and outliers were detected with respect to their API usage. They used the CHABADA prototype to carry out the behavior matching activity. 

In our work we carry out app metadata analysis and detect apps with certain vulnerabilities as per our heuristics. Our work focuses on generating rules or recommendations that are based on heuristics. Our system is capable of incorporating the results of all these studies as heuristics to detect malicious behavior in addition to our own heuristics.

In a study carried out by Huckvale et. al.~\cite{raey} it was found that despite the tests carried out by National Health Service (England) to ensure clinical data safety standards, apps had flouted privacy standards and sent data without encryption. This is the reason we are focusing on the loophole in content provider permission and the lack of any standard mechanism to verify whether a requester has access permissions or not. Expecting that the app developer would ensure security and leaving this loophole in android exposes users' data to potential attacks.

At present our work focuses on apps from the official Google Play Store. We intend to include other app stores too in the future as done in the work done by Lindorfer et.al.~\cite{Lindorfer2014AndRadar} at the International Secure System Lab. The ADMIRE system created by them let's you analyze different Android marketplaces, searching potentially malicious applications. This project also provides a score to various app markets using the reaction time to deletion and other factors when a new malware app shows up on their site. 

There have been multiple attempts at achieving the goal of properly managing access control on mobile (Android) devices. Efforts have been made by the open source community through the XPrivacy project (needs a rooted phone), the Privacy Guard project (available on Cyanogenmod, a custom Android ROM), the PDroid application (needs a rooted device). Research project by Conti et. al.~\cite{conti2011crepe} (CRePe), Enck et al.~\cite{enck2010taintdroid} (TaintDroid) and Jagtap et al.~\cite{Jagtap2011Privacy} (Preserving Privacy in Context-Aware Systems) have made similar efforts. CRePE described a system where security policy enforcement was carried out based on context of the smart phone. TaintDroid was a research effort where the data flow on an Android device was studied to figure out when sensitive data left the system via an untrusted application. The work of Jagtap et al.~\cite{Jagtap2011Privacy} focused on constraining data flow in a context-aware system using a policy-based framework. A related work by Ghosh et al.~\cite{ghosh2012privacy} used a similar policy driven approach to constrain application permissions based on context. In our work we are generating recommendations by studying app metadata and apks. We allow system administrators to choose from these recommendations and possibly make them part of their corporate policy.