\section{Vulnerability Description}
\label{vuln}
\noindent
When it comes to content providers, Google's Android documentation
describes two possible scenarios. The first states that data from
a provider
that specifies no permissions should not be accessible from other apps.
\begin{itemize}
 \item ``[...] If a provider's application doesn't specify any permissions, then other applications have no access to the provider's data. However, components in the provider's application always have full read and write access, regardless of the specified permissions.''~\footnote{Error in specification~\url{http://developer.android.com/guide/topics/providers/content-provider-basics.html#Permissions}}
 \item ``[...] All applications can read from or write to your provider, even if the underlying data is private, because by default your provider does not have permissions set. To change this, set permissions for your provider in your manifest file, using attributes or child elements of the <provider> element. You can set permissions that apply to the entire provider, or to certain tables, or even to certain records, or all three.''~\footnote{Installing permissions~\url{http://developer.android.com/guide/topics/providers/content-provider-creating.html#Permissions}}
\end{itemize}

Unfortunately, we found the first statement to be untrue. The following table shows that when a content provider app is not associating a permission to its provider then we have data leakage. This happens because Android does not verify that each provider has an associated permission. Possible solutions to mitigate this issue would either require a change in how Android handles content provider access control or a change in the app developer's code.

\begin{center}
	\begin{tabular}{ | p{2.5cm} | p{2.5cm} | p{2cm} | }
		\hline
		\textbf{Content Provider app} & \textbf{Content accessing app} & \textbf{Remark} \\
		\hline \hline
		No permission associated with provider & No permission used & \textcolor[rgb]{0.55,0,0}{\textbf{Potential data leakage}} \\
		\hline
		Permission associated with provider & No permission used & Permission denied \\
		\hline
		Permission associated with provider & Permission used & \textcolor[rgb]{0,0.33,0}{Ideal scenario} \\
		\hline
		No permission associated with provider & Permission used & Instillation error \\
		\hline
	\end{tabular}
\end{center}
